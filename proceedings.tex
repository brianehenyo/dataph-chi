\documentclass{sigchi}

% Use this section to set the ACM copyright statement (e.g. for
% preprints).  Consult the conference website for the camera-ready
% copyright statement.

% Copyright
\CopyrightYear{2020}
%\setcopyright{acmcopyright}
\setcopyright{acmlicensed}
%\setcopyright{rightsretained}
%\setcopyright{usgov}
%\setcopyright{usgovmixed}
%\setcopyright{cagov}
%\setcopyright{cagovmixed}
% DOI
\doi{https://doi.org/10.1145/3313831.XXXXXXX}
% ISBN
\isbn{978-1-4503-6708-0/20/04}
%Conference
\conferenceinfo{CHI'20,}{April  25--30, 2020, Honolulu, HI, USA}
%Price
\acmPrice{\$15.00}

% Use this command to override the default ACM copyright statement
% (e.g. for preprints).  Consult the conference website for the
% camera-ready copyright statement.

%% HOW TO OVERRIDE THE DEFAULT COPYRIGHT STRIP --
%% Please note you need to make sure the copy for your specific
%% license is used here!
% \toappear{
% Permission to make digital or hard copies of all or part of this work
% for personal or classroom use is granted without fee provided that
% copies are not made or distributed for profit or commercial advantage
% and that copies bear this notice and the full citation on the first
% page. Copyrights for components of this work owned by others than ACM
% must be honored. Abstracting with credit is permitted. To copy
% otherwise, or republish, to post on servers or to redistribute to
% lists, requires prior specific permission and/or a fee. Request
% permissions from \href{mailto:Permissions@acm.org}{Permissions@acm.org}. \\
% \emph{CHI '16},  May 07--12, 2016, San Jose, CA, USA \\
% ACM xxx-x-xxxx-xxxx-x/xx/xx\ldots \$15.00 \\
% DOI: \url{http://dx.doi.org/xx.xxxx/xxxxxxx.xxxxxxx}
% }

% Arabic page numbers for submission.  Remove this line to eliminate
% page numbers for the camera ready copy
% \pagenumbering{arabic}

% Load basic packages
\usepackage{balance}       % to better equalize the last page
\usepackage{graphics}      % for EPS, load graphicx instead 
\usepackage[T1]{fontenc}   % for umlauts and other diaeresis
\usepackage{txfonts}
\usepackage{mathptmx}
\usepackage[pdflang={en-US},pdftex]{hyperref}
\usepackage{color}
\usepackage{booktabs}
\usepackage{textcomp}

% Some optional stuff you might like/need.
\usepackage{microtype}        % Improved Tracking and Kerning
% \usepackage[all]{hypcap}    % Fixes bug in hyperref caption linking
\usepackage{ccicons}          % Cite your images correctly!
% \usepackage[utf8]{inputenc} % for a UTF8 editor only

% If you want to use todo notes, marginpars etc. during creation of
% your draft document, you have to enable the "chi_draft" option for
% the document class. To do this, change the very first line to:
% "\documentclass[chi_draft]{sigchi}". You can then place todo notes
% by using the "\todo{...}"  command. Make sure to disable the draft
% option again before submitting your final document.
\usepackage{todonotes}

% Paper metadata (use plain text, for PDF inclusion and later
% re-using, if desired).  Use \emtpyauthor when submitting for review
% so you remain anonymous.
\def\plaintitle{Exploring the Potential of Open Data Portals to Address the Information Needs of Lay Citizens}
\def\plainauthor{First Author, Second Author, Third Author,
  Fourth Author, Fifth Author, Sixth Author}
\def\emptyauthor{}
\def\plainkeywords{Information Seeking \& Search; Usability Study; Open data; Open government; HCI for Development; Lay citizens}

% llt: Define a global style for URLs, rather that the default one
\makeatletter
\def\url@leostyle{%
  \@ifundefined{selectfont}{
    \def\UrlFont{\sf}
  }{
    \def\UrlFont{\small\bf\ttfamily}
  }}
\makeatother
\urlstyle{leo}

% To make various LaTeX processors do the right thing with page size.
\def\pprw{8.5in}
\def\pprh{11in}
\special{papersize=\pprw,\pprh}
\setlength{\paperwidth}{\pprw}
\setlength{\paperheight}{\pprh}
\setlength{\pdfpagewidth}{\pprw}
\setlength{\pdfpageheight}{\pprh}

% Make sure hyperref comes last of your loaded packages, to give it a
% fighting chance of not being over-written, since its job is to
% redefine many LaTeX commands.
\definecolor{linkColor}{RGB}{6,125,233}
\hypersetup{%
  pdftitle={\plaintitle},
% Use \plainauthor for final version.
%  pdfauthor={\plainauthor},
  pdfauthor={\emptyauthor},
  pdfkeywords={\plainkeywords},
  pdfdisplaydoctitle=true, % For Accessibility
  bookmarksnumbered,
  pdfstartview={FitH},
  colorlinks,
  citecolor=black,
  filecolor=black,
  linkcolor=black,
  urlcolor=linkColor,
  breaklinks=true,
  hypertexnames=false
}

% create a shortcut to typeset table headings
% \newcommand\tabhead[1]{\small\textbf{#1}}

% End of preamble. Here it comes the document.
\begin{document}

\title{\plaintitle}

\numberofauthors{7}
\author{%
  \alignauthor{Leave Authors Anonymous\\
    \affaddr{for Submission}\\
    \affaddr{City, Country}\\
    \email{e-mail address}}\\
  \alignauthor{Leave Authors Anonymous\\
    \affaddr{for Submission}\\
    \affaddr{City, Country}\\
    \email{e-mail address}}\\
  \alignauthor{Leave Authors Anonymous\\
    \affaddr{for Submission}\\
    \affaddr{City, Country}\\
    \email{e-mail address}}\\
  \alignauthor{Leave Authors Anonymous\\
    \affaddr{for Submission}\\
    \affaddr{City, Country}\\
    \email{e-mail address}}\\
  \alignauthor{Leave Authors Anonymous\\
    \affaddr{for Submission}\\
    \affaddr{City, Country}\\
    \email{e-mail address}}\\
  \alignauthor{Leave Authors Anonymous\\
    \affaddr{for Submission}\\
    \affaddr{City, Country}\\
    \email{e-mail address}}\\
  \alignauthor{Leave Authors Anonymous\\
    \affaddr{for Submission}\\
    \affaddr{City, Country}\\
    \affaddr{for Submission}\\
    \affaddr{City, Country}\\
    \email{e-mail address}}\\
}

\maketitle

\begin{abstract}
Open government data allows for transparency from governments and access to data collected about its citizens. However, we are still far from achieving universal participation because data literacy and experience are necessary to extract insights from data. There is also no guarantee if available data can address people’s information needs. We explored the potential of open government data portals in addressing the information needs of lay citizens through an online survey and found that these can only be partially answered by the available data. We then conducted usability tests of the open data portals to understand their information seeking behavior, and semi-structured interviews to identify gaps in the design. We found that search engines are the preferred tool for seeking information and that citizens prefer visualized and processed data over raw data, which these portals lack. We conclude with design guidelines for open data portals catered to lay citizens.
\end{abstract}


% ACM Classfication

\begin{CCSXML}
<ccs2012>
<concept>
<concept_id>10003120.10003121</concept_id>
<concept_desc>Human-centered computing~Human computer interaction (HCI)</concept_desc>
<concept_significance>500</concept_significance>
</concept>
<concept>
<concept_id>10003120.10003121.10003125.10011752</concept_id>
<concept_desc>Human-centered computing~Haptic devices</concept_desc>
<concept_significance>300</concept_significance>
</concept>
<concept>
<concept_id>10003120.10003121.10003122.10003334</concept_id>
<concept_desc>Human-centered computing~User studies</concept_desc>
<concept_significance>100</concept_significance>
</concept>
</ccs2012>
\end{CCSXML}

\ccsdesc[500]{Human-centered computing~Human computer interaction (HCI)}
\ccsdesc[300]{Human-centered computing~Haptic devices}
\ccsdesc[100]{Human-centered computing~User studies}

% Author Keywords
\keywords{\plainkeywords}

% Print the classficiation codes
\printccsdesc
Please use the 2012 Classifiers and see this link to embed them in the text: \url{https://dl.acm.org/ccs/ccs_flat.cfm}

\section{Introduction}
The term "\textit{open government}" was first used in the 1950s and was generally used to refer to making previously confidential government information public with the goal of having a government be more politically transparent \cite{YuRobinson2012}. With the advancement of technology and the internet, governments started to launch open government data (OGD) portals as part of their involvement in open government initiatives such as Open Government Partnership \cite{DAWES201615, YuRobinson2012}.

By definition, \textit{open data} should be free to use, share, modify and analyze in machine-readable formats by anyone, but most especially the country's citizens \cite{corrales2019knowledge, DAWES201615}. Open data is perceived to be a useful tool for citizens to demand a more open and collaborative government \cite{warwick2017}. However, despite the initiative of the United Nations to provide guidelines for transparency and citizen engagement \cite{UN2013}, a gap still exists in utilizing OGD for such. It has been argued that most OGD users are technically skilled in analyzing data in its raw formats \cite{DAWES201615}. 

Unlike the open data movement, lobbyists of freedom of information (FOI) seeks for governments to establish constitutional guarantees for a "right-to-know" process. In the absence of publicly available government data, a passed FOI law ensures a data requester that they will get a response within a specified number of days. However, it is still the discretion of the government agency whether they will provide the data or not. 

Open Data Philippines (ODPH)\footnote{\url{https://data.gov.ph/}} was launched in January 2014 in fulfilment to the commitments of the Philippines to provide open data \cite{ODPH2015}. Despite the extensive usage of civil society organizations and the private sector, OGD in the Philippines still suffers from low utilization which suggests a disconnect between the published data and the information needs of the citizens \cite{galindesZaplan2018}. In 2016, an executive order from the President of the Philippines, Rodrigo Duterte, made the executive branch of the government comply with the people's rights to information \cite{duterte2016}. With this, the Electonic Freedom of Information (eFOI) portal\footnote{\url{https://foi.gov.ph}} was launched on November 2016 \cite{cigaral2016}. However, an FOI bill has yet to be passed into law. 

In governments that have embraced the benefits of OGD, there have been numerous data exploration applications developed through the years. In New York City, the Department of City Planning have different supplementary planning tools that utilize the data that the city has\footnote{\url{https://labs.planning.nyc.gov/projects/}}. In the UK, an open source web tool called Data:In Place\footnote{\url{https://data-in.place/}} was developed to help citizens utilize and understand open government data \cite{puussaar2018making}. Even in other developing countries around the globe, various case studies have been conducted to showcase the potential of open data in these countries \cite{farahi2018}. Although the impact was not quantified, it is suggested that further research may be done to evaluate the effects of OGD in developing countries \cite{farahi2018}.

These tools are examples of what can be done to bridge the gap in citizen engagement with OGD. However, it is necessary first to understand how ordinary citizens interact with data. Given the opportunity for human-computer interaction (HCI) to contribute to this, specifically in the context of a developing country like the Philippines, we are hoping to answer the following research questions:
\begin{itemize}
    \item RQ1: Do existing open government data portals support the information needs of the citizens? Why or why not?
    \item RQ2: Where and how do citizens look for information about the government?
    \item RQ4: How effective are Open Data Philippines and Freedom of Information portals in supporting the information seeking and sensemaking needs of the citizens?
\end{itemize}

To answer the first two research questions, we conducted a survey to explore the awareness and interest of Filipino citizens with regard to open government data. To answer the third and fourth research questions, we conducted semi-structured interviews with 21 individuals to observe their information seeking and sensemaking behavior through predetermined tasks. Participants were recruited based on convenience but priority was given to ordinary citizens with little to no experience on data analysis.

We found that ...

This study serves as our formative research in developing a web support tool to better communicate open government data to Filipino citizens.


\section{Related Work}
A survey of research in open government data implementation and human-data interaction (HDI) suggest the need for insights from ordinary citizens since they are also the target users of OGD portals \cite{warwick2017}. While existing HCI and HDI research provide ways of improving OGD portals for experienced data users, understanding behavior of ordinary citizens in contrast with data users present an opportunity to better design OGD portals and other supplementary tools. 

\subsection{OGD Portal Evaluation}
Most research conducted around OGD portals focused on implementation of developed countries \cite{kacprzak2019characterising, klimek2019dcat, koesten2019collaborative,Parycek2014}. Using the Global Open Data Index (GODI) as a reference, countries that have been evaluated in these studies rank relatively high in the list of the 94 countries (See Table \ref{tab:godi})\cite{godimetric2016}. This index only measures the openness of data published according to guidelines under \textit{Open Definition} \cite{godimetric2016}. 

Kl{\'\i}mek focused on evaluating how open data should be published to promote usability, portability, availability and performance \cite{klimek2019dcat}. It showed the value of having a standardized format in publishing metadata to supplement work on Linked Open Data. Parycek et al. \cite{Parycek2014} evaluated Vienna's OGD implementation through understanding the perspectives from both internal and external stakeholders in terms of the process, requirements and benefits. These evaluations are highly beneficial to the government stakeholders to help in improving the implementation of OGD portals as a whole. Our study is similar to \cite{Parycek2014} in the aspect of understanding the targeted end users, specifically evaluating how ordinary citizens perceive ODG portals.

\begin{table}
    \centering
    \begin{tabular}{c c c}
         \textbf{Rank} & \textbf{Country} & \textbf{GODI Score} \\
         \midrule
         2 & Great Britain & 79\% \\
         2 & Australia & 79\% \\
         5 & Canada & 69\% \\
         11 & United States & 65\% \\
         27 & Czech Republic & 50\% \\
         28 & Austria & 49\% \\ 
        %  58 & Philippines & 30\% \\
    \end{tabular}
    \caption{Global Open Data Index (GODI) ranking of countries evaluated by previous research with the Philippines for reference}
    \label{tab:godi}
\end{table}

\subsection{OGD User Studies}
User studies surrounding OGD portals have been focused on individuals who understand how to use data since these are the main users of current OGD portals \cite{choi2017characteristics,kacprzak2019characterising, koesten2019collaborative, koesten2017trials}. Despite the users' experience working with data, current OGD portals have design and implementation limitations in terms of functionalities for search \cite{kacprzak2019characterising, koesten2017trials} and collaborative work \cite{choi2017characteristics, koesten2019collaborative}.

A suggested framework on how data workers interact with data show that behaviors are dependent on the requirements of the task \cite{koesten2017trials}. Despite the existence of data portals, majority of the participants reported using Google to find data needed \cite{koesten2017trials}. To understand further the information seeking behavior of the users, dataset search queries were analyzed together with crowd sourced queries \cite{kacprzak2019characterising}. These studies serve as our basis for conducting experiments with ordinary citizens as the users of an OGD portal. 

\begin{comment}
Collaborative work using OGD was found to be interdisciplinary in nature \cite{choi2017characteristics}. Even though the focus of our study is not on collaborative data work, it is worth noting that the typical projects that emerge from OGD are statistical analysis and exploratory data tools that allow end users to examine data on their own \cite{choi2017characteristics}. 
\end{comment}

\begin{comment}
We conducted a climate survey to gauge the awareness and the interest of Filipino citizens about open government data. We also conducted a pilot study to evaluate the usability of the Open Data Philippines and Freedom of Information portals through search tasks. The survey was distributed electronically through snowball sampling on July 2019 while the usability interviews were conducted between July to August 2019 around the areas of Metro Manila. We obtained participant consent as per research ethics requirements to be able to conduct this study.
\end{comment}

\section{Climate Survey and Portal Analysis}
% rationale for conducting this first study
We conducted a 15 to 20 minute climate survey to gauge the awareness and the interest of Filipinos about the open government initiatives of the country, specifically ODPH and eFOI. We also conducted an inventory analysis of the available data in ODPH. The survey was released in July 2019 through Google Forms. Participant consent was acquired prior to proceeding with the questionnaire as per institutional ethical research requirements.

\begin{table}[t]
    \centering
    \begin{tabular}{|c|p{5.5cm}|}
         \hline
         \textbf{Level} & \textbf{Scenario}  \\
         \hline
         Personal &  Suppose you want to live in a new apartment in Makati City, and you want to know how safe the area is before making a decision.\\
         \hline
         Extrapersonal & Suppose you or your child wants to study in a state university within Metro Manila. Before making a decision, you want to know which state university has an acceptable faculty/student ratio.\\
         \hline
         Impersonal & Suppose you want to know where most Philippine presidents go for their foreign trips.\\
         \hline
    \end{tabular}
    \caption{Information seeking scenarios}
    \label{tab:scenarios}
\end{table}

\subsection{Survey Design}
We designed our survey to be as extensive as possible in understanding the factors that could potentially drive the awareness of Filipinos about open government initiatives. Prior to asking about their awareness of ODPH and eFOI, we asked them to provide a background of their data work experience and their information source preference. 

For data analysis experience, we asked our participants where they had used data before. Choices provided were for their main occupation, as a paid part-time job, as a hobby, as a passion or advocacy, for curiosity, and a final option of any other type of work. Based on a previous study, most data analysis work done are descriptive and exploratory and less inferential and predictive\cite{choi2017characteristics}. Using this information, we also asked how often they performed descriptive, diagnostic,  prescriptive, and predictive analyses with their data for each type of data work they did.

Seeking to understand the information seeking behavior of Filipinos, we provided three scenarios with different degrees of personal involvement in decision making. Table \ref{tab:scenarios} lists the specific scenario we provided in the survey. The personal scenario involves making a decision involving oneself. The extrapersonal scenarios may affect the individual to some extent but is more focused on another person. Lastly, the impersonal scenario involves looking for information that is more detached from the personal lives. We asked them which information sources they would most likely consult in making the decision.

Finally, we provided a brief description of ODPH and eFOI prior to asking if they have ever heard or used these two portals. We used a 5-point Likert scale to measure the frequency and usefulness of the portals for those who have at least visited or utilized them. In the end, we asked them to provide their top three sectors of government they would be most interested in to help us determine whether the interests of the citizens align with what is provided by the government.

\subsection{Participants}
The only requirement to participate in our survey was to be a Filipino citizen. We utilized snowball sampling by distributing the survey through Facebook, Twitter and different Facebook Groups to further our reach. We also made a Tagalog version of the survey to serve the non-English speaking citizens of the country. We were able to collect a total of 119 valid responses from both forms with 6 of which coming from the Tagalog survey.

More than half (N=63) of our respondents were women. The average age was 33 (SD=13.77). Majority were students (N=30) followed by professionals in Information Technology (N=22) and Business Consulting and Management (N=15). However, we were able to get respondents from different sectors as to gain varied insights. Because of this, more than half (N=69) responded having done data work as part of their main occupation in the past 12 months. Some respondents (N=54) even reported doing data work in more than one of the categories provided.

\subsection{Portal Analysis}
% Paolo is writing here. How the Data Inventory was created. Fixing this part pa
We created our data inventory based on the number of datasets in the open government data portal. After manually inspecting the open government data portal, we found out that there are 273 datasets available. We then inspected each dataset to determine its spatial and temporal granularity of the inspected data, and save the results in a spreadsheet for further analysis.

% Kyle will write the results of the survey
\subsection{Results}
We start by discussing the unanticipated issues encountered during the analysis of our climate survey and the portal analysis. To address whether or not open government data portal serves the needs of lay citizens, we provide an understanding of the information needs of our respondents. We then use the results from the survey and the data inventory to answer our first research question.

\subsubsection{Unanticipated Issues}
We raise some unanticipated issues that could arise from a climate survey that targets any Filipino citizen. Through no control over the responses that will be provided by the citizens, several erroneous or ambiguous answers to the questions were given. These answers were taken and grouped during analysis to prevent any unneeded assumptions. However, our study prevents these groups from affecting the overall result of the climate survey.

\subsubsection{Understanding Citizen Information Needs}
To answer our first research question, we must first understand the information needs and level of awareness of the citizens. We defined the information needs as the interests of the citizen based on a list of categories taken from Open Data Philippines categories and awareness into 4 categories: (1) Fully aware and used the portal, (2) Aware and have visited the portal, (3) Has heard about the portal, and (4) Did not know about the portal. We found that most Filipino citizens are interested in Business and Economics (N = 62), Transport (N = 56), and Disaster and Rehabilitation / Environment (N = 45). These top categories could be chosen due to their effect on the personal lives of these citizens. Business and Economics is the closest category related to the financial concerns, Transportation is a main concern of citizens for people living in highly populated areas such as Metro Manila, and Disaster and Rehabilitation / Environment since the Philippines is very disaster prone country due to prominent flooding and being near the pacific ring of fire. With regards to the awareness of these open data portals, we found that people are generally more aware of Open Data Philippines rather than Freedom of Information. 

% \subsubsection{RQ1: Efficacy of Open Government Data}
% Changed portals to data instead since this is focused highly on the data available instead of the portals

\subsubsection{Available Data}
\textit{\textcolor{red}{Discuss here the statistics of our data inventory and how much data is available with regard to the top interests of the survey respondents.}}

In addition to the categories provided, we also asked our respondents what other relevant questions they might have for the next 12 months. Since this was a free-form text field, we manually classified the questions according to the themes that emerged. The most common query was deciding where to move (N=18). Transportation related queries (N=15) involved travel routes, improvement to public transportation and alternative transportation development such as bike lanes. Economic queries (N=10) revolved around inflation and the overall economic development of the country. 

After categorizing each query provided, we individually assessed if they can be answered by existing Philippine OGD. 22.43\% of the queries may potentially be answered in part or whole by available datasets in the ODPH, while 24.30\% may be found in existing requests in eFOI. Most queries would require a combination of datasets from various sources before it may be answered.

\begin{table*}[t]
  \centering
  \begin{tabular}{l p{5.2cm} p{5.2cm}}
    % \toprule
    {\small\textit{Portal}}
    & {\small \textit{Personal Task}}
      & {\small \textit{Impersonal Task}}\\
    \midrule
    Open Search & How prevalent is crime in the province where you live? & Find the number of students enrolled in the University of the Philippines during 2015.\\
    Open Data Philippines & How much, in Philippine Peso,  was lost due to fire in your city of residence in 2013? & What are the top countries that most Philippine presidents go to as part of their work? \\
    Freedom of Information & How many people have work in your region? & How many road accidents happened in Metro Manila in 2018?
    % \bottomrule
  \end{tabular}
  \caption{List of search tasks performed by the participants per portal.}~\label{tab:searchtasks}
\end{table*}

\section{Usability Experiment}
To further examine the information seeking behavior of lay citizens, we conducted usability tests of ODPH and eFOI. It was done in parallel with semi-structured interviews to identify gaps in the design of the portals. The interviews were conducted between July and August 2019 at locations convenient to the participants. Participant consent was obtained prior to conducting the interview. Dialogues were audio recorded while the on-screen behavior was recorded using a screen capture software. 

\subsection{Participants}
For this study, our participants were selected based on their availability to be interviewed during the study period. We focused on recruiting individuals with (1) less than a year of experience with data analysis and (2) no experience using open government data portals for any kind of work. For participants that meet either criteria, they are classified as a \textit{lay person}. Otherwise, we call them \textit{data workers}. We were able to successfully recruit 13 lay people and 8 data workers for our interview, however, 2 of the lay people expressed their lack of experience working with computers in general. 

Majority of our participants were men (N=14). The average age for all participants was 31 (SD=15.08). The average age for both groups remained at 31 although our lay respondents had a slightly larger deviation (SD=16.36) than the data workers (SD=13.81). \textcolor{red}{Reference to the table of participants here?}

\subsection{Experiment Setup}
Each participant, regardless of classification, was asked to perform the same number of search tasks. There were three parts with two tasks per part. Table \ref{tab:searchtasks} shows the specific questions that were asked. 

In the open search, participants were allowed to use their preferred search engine, while in the other two parts, they were restricted to search only within the mentioned portal. Tasks were designed based on the assumption that the information is available in either a dataset from ODPH or a request in eFOI. The logic behind identifying the participant's place of residence as part of the personal task is based on a previous study \cite{puussaar2018making} that show people associate better with data when it is linked to a place.

All participants were timed for each task to quantify how long it took for them to complete the task. Each participant was given a 15-minute time limit to complete each task due to the constraint of the screen recording software used. They were also given the option to give up at any point. After each task, participants were asked about their experience and their thought process of how they approached the problem.

We also provided the participants six different datasets downloaded from the two government portals for them to rank their usability and provide the basis for their ranking. The goal of this task is to understand the thought process of ordinary citizens and the factors they consider when presented with raw datasets instead of organized information. 

\subsection{Issues  Encountered}

\subsection{Data Analysis}
Inductive coding - Exploratory in nature
\url{https://medium.com/@projectux/themes-dont-just-emerge-coding-the-qualitative-data-95aff874fdce}

descriptive coding - summarizes the primary topic
process coding - word or phrase that captures the action
in vivo coding - using the participant's own words

structural coding is a question-based code that acts as a labelling and index device, allowing researchers to quickly access data likely to be relevant to a particular analysis from a larger data set. Its used as a categorisation technique for further qualitative data analysis.

We have a mix of quantitative and qualitative data from the survey responses, field notes from the interviews, interview transcripts and screen recordings. For the qualitative data, we used topic coding to categorize the responses according to the research questions.

Understanding the level of awareness of Filipino citizens about Open Data Philippines and Freedom of Information (RQ1) and their interest in open government data (RQ2) required processing and analyzing the responses from the survey which were quantitative in nature because of the Likert-scale responses. For the free form wild card questions on interests not included in the predefined options, we used thematic analysis to group together responses.

To understand the information seeking behavior of Filipinos (RQ3), we took the screen recordings and analyzed the different websites the participants used to find answers to the tasks presented. We also analyzed the search queries used per task to come up with the most common used keywords. In addition, we also analyzed data from the information seeking section of our survey.

In analyzing the queries and search behavior of the interviewees, keywords were categorized by whether they were spatial, temporal, or special keywords. Special keywords include words or phrases not explicitly stated in the task which pertain to neither temporal nor spatial information but could still aid in the completion of the task. Special keywords include specific news websites, government agencies responsible for collecting said data, and specific file formats. The number of queries used per task and the number of queries that were posed in a question format were also analyzed. The success rate of interview tasks was also compared across the types of task done and across the classifications of the interviewees. To deepen our understanding of the success rate of tasks, a comparison was done to see the average time taken for successful and failed tasks across different types of tasks and across the classifications of the interviewees.

Lastly, to evaluate the efficacy of the Open Data Philippines and Freedom of Information portals (RQ4), we analyzed the responses of the interviewees to identify the features that supported and restricted them from completing the tasks. 


\textit{Where do the speed of finding the responses come in?}

\subsection{Results}

% RQ2: Where and how do citizens look for information?

\subsubsection{Query Analysis}
Occurrence of search words used per task were analyzed for the purpose of determining commonly used words and possible search behaviors. For tasks that dealt with searching for information related to a specific time, temporal keywords were predominantly used to filter the results. For all of the tasks, interviewees primarily made use of keywords presented in the task itself with minimal deviation or use of special keywords that could help in their search. This indicates a more straightforward behavior of individuals when searching for information on both their preferred search engine and on open data portals.

In analysing the frequency of use of specific keywords, it was found that data workers made use of more special keywords to aid in their search. This allowed for a more flexible search as they knew how to specify certain agencies and file formats to filter their results. Queries were also rarely posed in a question format while spatial and temporal keywords were more often used by all the interviewees. The use of queries in a question format was evident only in lay people who assumed that doing so would yield more direct search results.

% RQ3: How effective are Open Data Philippines and Freedom of Information portals in supporting the information seeking and sensemaking needs of the citizens?

\subsubsection{Platform Usability in Accomplishing the Tasks}

\section{Discussion}
\textit{Comparative discussion of both survey and interviews.}

\begin{comment}
From model paper:
We address our first two research questions by summarizing our results. We then discuss insights to broader policy implications to improve the feasibility of online grocery delivery services within low-income and transportation-scarce regions. We conclude by contributing design implications for how online-grocery service interfaces can address participant barriers to using such services. To answer our fnal research question, we leverage past HCI literature to support our implications and further extend the literature.
\end{comment}


\section{Limitations}
- A more statistical approach for sampling for the survey to get a better representation of the population

\section{Conclusion and Future Work}
Summary of what we did and what we found out

We contribute design guidelines that could help improve universal participation of lay people in utilizing open government data.

Moving forward, we would like to understand whether our design suggestions facilitate a better way to communicate open government data to the general public, even to those with minimal data literacy. 

\section{Acknowledgments}

Sample text: We thank all the volunteers, and all publications support
and staff, who wrote and provided helpful comments on previous
versions of this document. Authors 1, 2, and 3 gratefully acknowledge
the grant from NSF (\#1234--2012--ABC). \textit{This whole paragraph is
  just an example.}
  


% Balancing columns in a ref list is a bit of a pain because you
% either use a hack like flushend or balance, or manually insert
% a column break.  http://www.tex.ac.uk/cgi-bin/texfaq2html?label=balance
% multicols doesn't work because we're already in two-column mode,
% and flushend isn't awesome, so I choose balance.  See this
% for more info: http://cs.brown.edu/system/software/latex/doc/balance.pdf
%
% Note that in a perfect world balance wants to be in the first
% column of the last page.
%
% If balance doesn't work for you, you can remove that and
% hard-code a column break into the bbl file right before you
% submit:
%
% http://stackoverflow.com/questions/2149854/how-to-manually-equalize-columns-
% in-an-ieee-paper-if-using-bibtex
%
% Or, just remove \balance and give up on balancing the last page.
%
\balance{}
% BALANCE COLUMNS
\balance{}

% REFERENCES FORMAT
% References must be the same font size as other body text.
\bibliographystyle{SIGCHI-Reference-Format}
\bibliography{sample}

\end{document}

%%% Local Variables:
%%% mode: latex
%%% TeX-master: t
%%% End:
